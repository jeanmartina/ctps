 \documentclass[a4paper,10pt]{article}

\usepackage{lmodern}
\usepackage[utf8]{inputenc}
\usepackage[T1]{fontenc}
\usepackage[brazilian]{babel}
\usepackage[margin=0.5in]{geometry}
\usepackage{booktabs}
\usepackage{tablefootnote}
\usepackage{tabularx}
\usepackage{color}    % Controle das cores
\usepackage[colorlinks=true]{hyperref}
\usepackage{graphicx}
\usepackage{amsmath}
\usepackage{amsfonts}
\usepackage{enumitem}
\usepackage{multicol}

\author{Makhles Reuter Lange \\
        Ciências da Computação - CCO \\
        Departamento de Informática e Estatística - INE \\
        Universidade Federal de Santa Catarina - UFSC}
\title{Carteira de Trabalho em Blockchain}
\date{Florianópolis, junho de 2018}

\begin{document}
\maketitle

%------------------------------------------------------------------------------
%------------------------------------------------------------------------------
\section{Introdução}

Este trabalho consiste no desenvolvimento de uma nova solução para Carteiras de Trabalho utilizando a tecnologia \emph{blockchains}. Será criado um contrato inteligente com o auxílio da plataforma Ethereum.

\section{Requisitos}\label{sec:req}
\subsection{Requisitos Funcionais}\label{sec:req_funcionais}

\begin{itemize}
  \item \textbf{\texttt{[RF01]}} Possibilitar a criação da carteira de trabalho de uma pessoa que forneça seus dados pessoais (vide~\texttt{RE02} na Seção~\ref{sec:req_externos}). 
  \item \textbf{\texttt{[RF02]}} Permitir que o empregado registre uma alteração no seu nome. Esta situação poderá ocorrer quando houver uma mudança no seu estado civil.
  \item \textbf{\texttt{[RF03]}} Permitir que o empregado registre uma alteração no seu estado civil.
  \item \textbf{\texttt{[RF04]}} Permitir que um empregador solicite a firma de um contrato de trabalho com o dono da carteira de trabalho.
  \item \textbf{\texttt{[RF05]}} Permitir que o dono da carteira de trabalho aceite a solicitação de firma de contrato de trabalho.
  \item \textbf{\texttt{[RF06]}} Possibilitar que o empregador e o empregado rescindam um contrato de trabalho firmado entre si.
  \item \textbf{\texttt{[RF07]}} Possibilitar que o empregador adicione licenças concedidas ao empregado. 
  \item \textbf{\texttt{[RF08]}} Permitir que o INSS adicione registros sobre afastamentos do empregado.
  \item \textbf{\texttt{[RF09]}} Permitir que o empregador adicione períodos de férias gozadas pelo empregado.
  \item \textbf{\texttt{[RF10]}} Permitir o cálculo do tempo de trabalho total desde a criação da carteira de trabalho.
\end{itemize}

\subsection{Requisitos Não-Funcionais}
\begin{itemize}
  \item \textbf{\texttt{[RNF01]}} A carteira de trabalho deverá ser feita utilizando contratos inteligentes.
  \item \textbf{\texttt{[RNF02]}} O contrato inteligente deverá ser feito utilizando a linguagem Solidity.
  \item \textbf{\texttt{[RNF03]}} O teste do contrato inteligente deverá ser feito utilizando a ferramenta \href{https://remix.ethereum.org/}{Remix}.
\end{itemize}

\subsection{Requisitos Externos}\label{sec:req_externos}
\begin{itemize}
  \item \textbf{\texttt{[RE01]}} O cálculo do tempo de trabalho, em relação aos requisitos funcionais \texttt{RF07}, \texttt{RF08} e \texttt{RF09}, deverá atender às normas legais definidas na Consolidação das Leis do Trabalho.
  \item \textbf{\texttt{[RE02]}} A carteira de trabalho desenvolvida não deverá expor as informações pessoais do seu dono.

  Atualmente, todos os contratos e dados contidos no blockchain podem ser vistos em bytecode por qualquer pessoa. Para impedir que os dados pessoais do dono da carteira de trabalho sejam visíveis, decidiu-se pela utilização do \emph{hash} destes dados, ao invés dos dados em si. Supõe-se, então, que haverá uma entidade reguladora responsável por emitir a carteira de trabalho para o cidadão, e que esta entidade, de posse dos seus dados, fornecerá o \emph{hash} dos dados na criação da carteira.

  \item \textbf{\texttt{[RE03]}} Caberá a esta entidade, também, definir os dados que o cidadão deverá fornecer para a criação de sua carteira. Cita-se, como exemplo:

  \begin{multicols}{3}
    \begin{itemize}
      \item Nome do cidadão;
      \item Filiação;
      \item Data de nascimento.
      \item Naturalidade;
      \item Sexo;
      \item Documento de identidade;
      \item Título de eleitor;
      \item PIS/PASEP;
      \item Estado civil;
      \item Número do CPF;
      \item Número da CNH;
      \item Grupo sangüíneo;
      \item Possui alergias?
      \item É hemofílico?
      \item É diabético?
      \item É doador de órgãos?
      % \item Documento de identidade:
      % \begin{itemize}
        % \item Número;
        % \item Data de emissão.
        % \item Órgão emissor;
      % \end{itemize}
      % \item Título de eleitor:
      % \begin{itemize}
        % \item Número;
        % \item Zona;
        % \item Seção;
      % \end{itemize}
    \end{itemize}
  \end{multicols}
\end{itemize}

\section{Critérios de Sucesso}

% Conforme estabelecido na Seção~\ref{sec:req}, este trabalho possui dez requisitos funcionais. 
Propõe-se que a avaliação seja feita com base no sucesso da execução de cada um requisitos funcionais definidos na Seção~\ref{sec:req_funcionais}, e no atendimento aos requisitos externos \texttt{RE01} e \texttt{RE02}, totalizando 12 requisitos. Neste caso:

\[ \text{Nota do trabalho } = \frac{\text{ número de requisitos atendidos }}{12} \]

\end{document}