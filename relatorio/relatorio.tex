 \documentclass[a4paper,10pt]{article}

\usepackage{lmodern}
\usepackage[utf8]{inputenc}
\usepackage[T1]{fontenc}
\usepackage[brazilian]{babel}
\usepackage[margin=0.5in]{geometry}
\usepackage{booktabs}
\usepackage{float}    % For floating images and tables
\usepackage{threeparttable}
\usepackage{tablefootnote}
\usepackage{tabularx}
\usepackage{color}    % Controle das cores
\usepackage{listings} % Para incluir código fonte com formatação
\usepackage{xfrac}
\usepackage[colorlinks=true]{hyperref}
\usepackage{graphicx}
\usepackage{amsmath}
\usepackage{amsfonts}
\usepackage{multirow}
\usepackage{xfrac}  % Slanted fractions

% ---
% Listings
% ---
\renewcommand{\lstlistingname}{Algoritmo}% Listing -> Algoritmo
\renewcommand{\lstlistlistingname}{Lista de \lstlistingname s}% List of Listings -> Lista de Algoritmos
\definecolor{mygreen}{rgb}{0,0.6,0}
\definecolor{mygray}{rgb}{0.5,0.5,0.5}
\definecolor{mymauve}{rgb}{0.58,0,0.82}
\lstset{ %
  backgroundcolor=\color{white},     % choose the background color;
  basicstyle=\footnotesize\ttfamily, % the size of the fonts that are used for the code
  breakatwhitespace=false,           % sets if automatic breaks should only happen at whitespace
  breaklines=true,                 % sets automatic line breaking
  captionpos=t,                    % sets the caption-position to bottom
  commentstyle=\color{mygreen},    % comment style
  deletekeywords={...},            % if you want to delete keywords from the given language
  escapeinside={\%*}{*)},          % if you want to add LaTeX within your code
  extendedchars=true,              % lets you use non-ASCII characters; for 8-bits encodings only, does not work with UTF-8
  frame=single,                    % adds a frame around the code
  keepspaces=true,                 % keeps spaces in text, useful for keeping indentation of code (possibly needs columns=flexible)
  keywordstyle=\color{blue},       % keyword style
  language=Python,                 % the language of the code
  morekeywords={*,...},            % if you want to add more keywords to the set
  numbers=none,                    % where to put the line-numbers; possible values are (none, left, right)
  numbersep=5pt,                   % how far the line-numbers are from the code
  numberstyle=\tiny\color{mygray}, % the style that is used for the line-numbers
  rulecolor=\color{black},         % if not set, the frame-color may be changed on line-breaks within not-black text (e.g. comments (green here))
  showspaces=false,                % show spaces everywhere adding particular underscores; it overrides 'showstringspaces'
  showstringspaces=false,          % underline spaces within strings only
  showtabs=false,                  % show tabs within strings adding particular underscores
  stepnumber=2,                    % the step between two line-numbers. If it's 1, each line will be numbered
  stringstyle=\color{mymauve},     % string literal style
  tabsize=2,                       % sets default tabsize to 2 spaces
  title=\lstname,                  % show the filename of files included with \lstinputlisting; also try caption instead of title
  numberbychapter=false            % number lists by chapter or sequentially from the beginning of the document
}



\author{Makhles Reuter Lange \\
        Ciências da Computação - CCO \\
        Departamento de Informática e Estatística - INE \\
        Universidade Federal de Santa Catarina - UFSC}
\title{Carteira de Trabalho em Blockchain}
\date{Florianópolis, junho de 2018}

\begin{document}
\maketitle

%------------------------------------------------------------------------------
%------------------------------------------------------------------------------
\section{Introdução}

Este trabalho consiste no desenvolvimento de uma nova solução para Carteiras de Trabalho utilizando a tecnologia \emph{blockchains}. Será criado um contrato inteligente com o auxílio da plataforma Ethereum.


\section{Requisitos Funcionais}

\begin{enumerate}
  % \item \textbf{Criar a carteira de trabalho} - Possibilitar que o usuário crie sua carteira de trabalho informando os dados obrigatórios: nome, filiação, data de nascimento, número do documento de identidade e órgão emissor, número do título de eleitor, PIS/PASEP, estado civil, naturalidade, sexo, CPF e CNH.
  \item \textbf{Criar a carteira de trabalho} - Possibilitar que o usuário crie a sua carteira de trabalho informando os seguintes dados pessoais obrigatórios (dados do tipo \texttt{string}, a não ser onde especificado):
    \begin{itemize}
      \item Nome;
      \item Filiação;
      \item Data de nascimento. Tipo: \texttt{date};
      \item Documento de identidade:
      \begin{itemize}
        \item Número;
        \item Data de emissão. Tipo: \texttt{date};
        \item Órgão emissor;
      \end{itemize}
      \item Título de eleitor:
      \begin{itemize}
        \item Número;
        \item Zona;
        \item Seção;
      \end{itemize}
      \item PIS/PASEP;
      \item Estado civil. Opções: \texttt{[ SO | C | D | SE | V ]};
      \item Naturalidade;
      \item Sexo. Opções: \texttt{[ M | F ]};
      \item Número do CPF;
      \item Número da CNH;
      \item Grupo sangüíneo. Opções: \texttt{[ OP | ON | AP | AN | BP | BN | ABP | ABN ]};
      \item Possui alergias? Opções: \texttt{[ S | N ]};
      \item É hemofílico? Opções: \texttt{[ S | N ]};
      \item É diabético? Opções: \texttt{[ S | N ]};
      \item É doador de órgãos? Opções: \texttt{[ S | N ]}.
    \end{itemize}
  \item \textbf{Alterar o nome} - O empregado poderá alterar o seu nome, contanto que forneça o motivo, \emph{e.g.}, matrimônio.
  \item \textbf{Alterar o estado civil} - O empregado poderá alterar o seu estado civil.
  \item \textbf{Solicitar a firma de um contrato de trabalho} - O empregado e o empregador poderão solicitar a firma de um contrato de trabalho entre si.
  \item \textbf{Firmar um contrato de trabalho} - Possibilitar que um contrato de trabalho entre o empregador e empregado seja firmado.
  \item \textbf{Rescindir um contrato de trabalho} - Possibilitar que o empregador e o empregado rescindam um contrato de trabalho.
  \item \textbf{Adicionar licença} - Possibilitar que o empregador adicione licenças concedidas ao empregado. 
  \item \textbf{Adicionar período de afastamento} - Possibilitar que o INSS faça anotações sobre afastamentos do empregado.
  \item \textbf{Adicionar período de férias} - Possibilitar que o empregador adicione períodos de férias gozadas pelo empregado.
  \item \textbf{Calcular tempo de trabalho} - Possibilitar o cálculo do tempo de trabalho total desde a criação da carteira de trabalho.
\end{enumerate}

\section{Requisitos Não-Funcionais}
\begin{enumerate}
  \item A carteira de trabalho deverá ser feita utilizando contratos inteligentes.
  \item O contrato inteligente deverá ser feito utilizando a linguagem Solidity.
  \item O teste do contrato inteligente deverá ser feito utilizando a ferramenta \href{https://remix.ethereum.org/}{Remix}.
\end{enumerate}
\section{Requisitos Externos}



\end{document}